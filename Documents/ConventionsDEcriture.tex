\documentclass[12pt,a4paper]{article}
\usepackage[utf8]{inputenc}
\usepackage[french]{babel}
\date{}
\title{Convention d'écriture et liste des procédures}
\begin{document}
	\maketitle
	Ce docuement sert à produire un code qui est lisible et correctible par l'ensemble de nous deux. Ce sont des règles de bienséance de code.\\
	 Le diagramme de clase devra \emph{\texttt{\underline{\textbf{TOUJOURS}}} être à jour}. Le document ici-présent sera mis-à-jour périodiquement vis-à-vis de celui-ci.
	\section{Convention d'écriture}
		\begin{description}
			\item[Les noms des fonctions et de variables] commenceront avec une minuscule et prendront une majuscule à chaque début de nouveau mot (e.g. \texttt{laPetiteFonction (Type laPetiteVariable, Type laPetiteVariablePrivee\_)})
			\item[Les noms de variables] seront explicites vis-à-vis de leurs contenus
			\item[Les noms des fonctions] seront explicites vis-à-vis de leurs utilisations
			\item[La définition des fonctions] commencera par un commentaire résumant en \emph{UNE} phrase : le fonctionnement de la fonction, ses arguments d'entrée et ses arguments de sortie.
			\item[Les commentaires] seront obligatoires pour les parties peu claires du code
			\item[Les champs (variables inclus dans les classes) privés] auront des "\_" à la fin de leurs noms (e.g. \texttt{Personne JoueurHumain.nom\_})
			\item[Un ficher source appelé \texttt{test.cpp}] sera utilisé pour tester le fonctionnement du code de \emph{CHAQUE} partie du code.
		\end{description}
		
	\section{Listes des procédures}
		\subsection{De la classe \texttt{Affichage}}
			\subsubsection{\texttt{actualiseGrille}}
			\subsubsection{\texttt{clic}}
				\paragraph{Prototype}
					int clic();
				\paragraph{Description}
					Renvoi le entiers de la position relative du clic.
		\subsection{De la classe \texttt{Jeu}}
			\subsubsection{\texttt{changerJoueur}}
			\subsubsection{\texttt{getGrille}}
		\subsection{De la classe \texttt{GrilleLogique}}
			\subsubsection{\texttt{afficher}}
			\subsubsection{\texttt{renvoiPositionBateau}}
		\subsection{De la classe \texttt{Personne}}
			\subsubsection{Construteur \texttt{Personne}}
			\subsubsection{\texttt{passerTourSuivant}}
			\subsubsection{\texttt{choixCible}}
		\subsection{De la classe \texttt{JoueurHumain}}
			\subsubsection{\texttt{choixCible}}
		\subsection{De la classe \texttt{JoueurOrdi}}
			\subsubsection{\texttt{choixCibleOrdi}}
		\subsection{De la classe \texttt{Bateau}}
			
		\subsection{De la classe \texttt{menuBatailleNavale}}
		\subsection{De la classe \texttt{Menu}}
		\subsection{De la classe \texttt{OptionMenu}}
	
\end{document}